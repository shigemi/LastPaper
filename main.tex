% 独自のコマンド

% ■ アブストラクト
%	¥begin{jabstract} 〜 ¥end{jabstract}	:日本語のアブストラクト
%	¥begin{eabstract} 〜 ¥end{eabstract}	:英語のアブストラクト

% ■ 謝辞
%	¥begin{acknowledgment} 〜 ¥end{acknowledgment}

% ■ 文献リスト
%	¥begin{bib}[100] 〜 ¥end{bib}

\long\def\comment#1{}

\newif\ifjapanese

\japanesetrue	% 論文全体を日本語で書く(英語で書くならコメントアウト)

\ifjapanese
  \documentclass[11pt]{jreport}
  \renewcommand{\bibname}{参考文献}
  \newcommand{\acknowledgmentname}{謝辞}
\else
  \documentclass[11pt]{report}
  \newcommand{\acknowledgmentname}{Acknowledgment}
\fi

\usepackage{thesis}
\usepackage{ascmac}
\usepackage{graphicx}
\usepackage{multirow}
\usepackage{url}
% \usepackage{ylab_thesis}

\bindermode	% バインダ用余白設定

% 日本語情報(必要なら)
\jclass	{卒業論文}							% 論文種別
\jtitle		{Android端末を用いたバランスボードシステムの提案}			% タイトル。改行する場合は\\を入れる
\juniv		{慶應義塾大学}						% 大学名
\jfaculty	{環境情報学部 環境情報学科}				% 学部、学科
\jauthor	{重田 瑞希}						% 著者
\jadvisor	{増井 俊之}{教授}					% 指導教官、形式は『{名前}{肩書}』
\jhyear	{25}								% 平成○年度
\jsyear	{2013}							% 西暦○年度
%\jkeyword	{\LaTeX、テンプレート、卒業論文}			% 論文のキーワード


\begin{document}

%\ifjapanese
%  \jmaketitle % 表紙(日本語)
%\else
%  \emaketitle % 表紙(英語)
%\fi

% ■ アブストラクトの出力 ■
%	◆書式:
%		begin{jabstract}〜end{jabstract}	:日本語のアブストラクト
%		begin{eabstract}〜end{eabstract}	:英語のアブストラクト
%		※ 不要ならばコマンドごと消せば出力されない。


¥chapter{概要}
%¥label{}

近年,スマートフォン端末を所持するユーザは増加しており ¥cite{hoge01}、スマートフォン端末に内蔵されたセンサを利用したアプリケーションやサービスも同様に、日々増え続けている。例えば、照度センサが周囲の明るさを検知することで、端末のディスプレイの照度レベルを自動で調節してくれるシステムや、睡眠時の体動を端末内の加速度センサが検知し、それを元にユーザの睡眠の質を解析し、ユーザに心地よい目覚めを提供してくれるといった睡眠アプリケーションなど、生活に密着したサービスやアプリケーションの需要は高まっている。

本研究では、このスマートフォン端末を用いた、新しいバランスボードのシステムの提案を行う。

	% アブストラクト。要独自コマンド、include先参照のこと

\tableofcontents	% 目次
\listoffigures		% 図目次

\pagenumbering{arabic}

\chapter{序論}
\label{chap:introduction}


\section{研究背景}

近年、ICT技術の発展に伴い「ユビキタス社会」が人々にますます浸透してきている。
先に挙げたスマートフォン端末の例だと、たった指先程度の大きさしかない数々のセンサが、人々の生活の手間は省き、生活における利便性を高めた。
スマートフォン以外の事例では、駅の改札にてカードを改札にタッチするだけで構内に入れ、オートチャージ機能により、かつての切符を買う手間を要さなくなった。子供の玩具においては、かつてはゲーム機を手に持ち、起用に指でコントローラーを操作して遊んでいたものを、今やラケットを持つかの様にゲーム機を握り、テレビの前で本当にテニスや卓球を行っているかのようにができるようになった。

これを実現させたものが、任天堂株式会社が開発した「Wii」シリーズである。中でも、「Wii Fit」というバランスボード型のタイプのものは、「ゲーム感覚でフィットネスをする」という新しい概念のもと生み出されてたものであり、この「ゲーム性とフィットネス性を兼ねた」新しいエンタテイメントとして、最近注目されている。

こうした背景には、近年、国民の運動・健康に対する意識が高まっている現状があるだろう。厚生労働省が行った2013年の国民健康・栄養調査によると、健康づくりのための身体活動や運動を実践している者の割合は39.4%であり、70歳位上では半数を超えているそうだ。特に、70歳以上では、男女とも「要介護とならないため、もしくは悪化させないため」と回答した者の割合が最も多かった。\cite{hoge02}

こうした健康志向の高まっている高齢者も顧客ターゲットに含めた事で、今までのテレビゲームになかった「子供からお年寄りまでの幅広い層のユーザ獲得」につながったという。

その一方で、ユーザビリティの問題点から、なかなか継続するのが困難であるという意見もある。これに関しては、行ったアンケート調査結果を後の章で紹介するが、Wii Fitをするためのセットアップに手間がかかるなど、ユーザからは様々な指摘が挙げられた。




\section{研究目的}

本研究の目的は、以上のユーザビリティの煩わしさをセンサを用いて解決し、高まる人々の健康意識にあやかって開発された「Wii Fit」のフューチャーワークとしても用いる事ができるシステムを提案することにある。

今回は、スマートフォン端末を使用したバランスボードシステムを実装した。



\section{本論分について}

本論分は以下のように構成されている。

第\ref{chap:02}章では、バランスボードについて述べる。バランスボードの需要や役割、関連したプロダクトの紹介や問題点などについてまとめる。

%第\ref{chap:03}章では、関連研究について述べる。

第\ref{chap:04}章では、要素技術について述べる。実装で使用したシステム・技術について説明する。

第\ref{chap:05}章では、実装内容について述べる。システム構成や実装方法、また用途例を紹介する。

第\ref{chap:06}章では、最後に本研究のまとめと、今後の応用例などを考察したい。

また最後に、付録として実装のソースコードを貼付ける。
	% 本文1
\chapter{バランスボードの意義}
\label{chap:02}

本章では、バランスボードの歴史と需要、また関連したプロダクトの紹介や、既存のバランスボードが抱える課題点について述べていきたい。


\section{歴史}

元々バランスボートというのは、ライフル射撃のナショナルチームでのトレーニングに採用されたのがきっかけだという。それをきっかけに、怪我の軽減や運動パフォーマンスの向上、リハビリやメンタルトレーニングといった面で各競技において効果が表れ、少しずつ注目されていった。 \cite{hoge03}

当初のタイプは縦長の板の裏側に支点となる突起がついた、いわゆるシンプルな型であったというが、現在は用途に応じて様々なタイプのものがある。それを下で紹介する。


\section{種類}

ここでは、現在あるバランスボードの種類をいくつか紹介していきたい。

\subsection{突起型バランスボード}

これが従来のタイプのものである。当初と同じ、板の裏に支点となる突起がついたもの(図\ref{fig:01})や、縦長の板状で支点となる突起が棒状になっていて転がしてバランスをとるタイプのもの(図\ref{fig:02})などがある。



\begin{figure}[htbp]
 \begin{minipage}{0.5\hsize}
  \begin{center}
   \fbox{\includegraphics[height=30mm]{image02.eps}}
  \end{center}
  \caption{タイプ}
  \label{fig:01}
 \end{minipage}
 \begin{minipage}{0.5\hsize}
  \begin{center}
   \fbox{\includegraphics[height=30mm]{image01.eps}}
  \end{center}
  \caption{縦長タイプ}
  \label{fig:02}
 \end{minipage}
\end{figure}



これらはいわゆる従来型のバランスボードで、医療機関でのリハビリテーションやスポーツ選手のトレーニングなどでよく使用される。

また、ボード状のものだけではなく、球状のタイプでその上に乗ってバランスをとるバランスボールや、不安定な平型のクッション状のものにのってバランスをとるものもある。


\subsection{バランスWiiボード}

これが先に例で挙げてきた、任天堂株式会社が開発した「Wii FIt」を使用する際に用いるものである。(図\ref{fig:03})


\begin{figure}[htbp]
    \begin{center}
       \fbox{\includegraphics[height=30mm]{image03.eps}}
    \end{center}
    \caption{バランスWiiボード}
    \label{fig:03}
\end{figure}



従来のバランスボードとは異なり、板を支える支点となる突起や棒はなく、水平な板状の型をしており、ボードに内蔵されたセンサがその板上でのユーザの体重移動を検知することで、ユーザのバランス状況を解析する仕組みである。ユーザはこの水平なボードの上に乗り、画面上で繰り広げられる綱渡りやスノーボードといった娯楽を、まるで実際に行っているかのような感覚で体重移動することで、画面上の自分が動いて楽しめる、という仕様になっている。

そうしたスポーツ娯楽を始め、他にもヨガやピラティスといった本格的なフィットネスも楽しめ、今では40種類以上のフィットネスを楽しむ事ができるという。 \cite{hoge04}



\subsection{トーニングシューズ}

バランスボードとは種類が少し異なるが、これはバランスボールと同じ効果が得られるようデザインされたもので、靴底や中敷きの形状・素材に工夫がなされている。通常のシューズより足への負荷を高められ、エクササイズ目的で開発された靴である。(図\ref{fig:04})


\begin{figure}[htbp]
    \begin{center}
       \fbox{\includegraphics[height=30mm]{image04.eps}}
    \end{center}
    \caption{トーニングシューズ}
    \label{fig:04}
\end{figure}


上の写真の靴(LAギア ウォーキングトーン)は、靴底が接地面積が小さくなるよう設計されており、左右方向はやや不安定になり、意識的に体幹でバランスをとるようデザインされたものである。 \cite{hoge05}

現在、様々なメーカーでこういったフィットネス効果目的の商品が開発されている。




\section{需要}

ここでは、社会的な需要や既存プロダクトの問題点などについて、実際に行ったアンケート調査をふまえて述べていきたい。

\subsection{社会的現状}

先に挙げた健康・運動意識調査にもあったように、特に近年の高齢者におけるフィットネスの意識は高く、子供からお年寄りまで幅広い世代のユーザをターゲットにした「Wii FIt」が注目されているという点からも、社会全体としてこうした運動・健康傾向が表れている印象を受ける。これ以外にも、トーニングシューズなどフィットネス効果を狙った様々なプロダクト・サービスが開発されているのも、人々の関心の高さの象徴であろう。

その一方で、社会的問題として、子供の運動能力の低下や高齢者の公共空間における事故の増加が挙げられる。

文部科学省が行った「体力・運動能力調査」によると、子供の体力・運動能力の低下は昭和60年ごろから現在まで低下傾向が続いており、その原因として、保護者をはじめとする国民の意識の中で、外遊びやスポーツの重要性を学力の状況と比べ軽視する傾向が進んだ事にあるとされている。\cite{hoge06} さらに、テレビゲームやポータブルゲームの流行により、子供達が外で遊ぶより室内で遊ぶことが多くなった現状もあげられる。

また、高齢者の事故の増加に関しては、消防庁が行った調査の統計結果によると、高齢者の事故がここ5年間で1万人以上増加しており、その約8割を占める原因が「転倒事故」だという。 \cite{hoge07}そして、これら転倒事故が起こってしまう要因として主にあげられるのが、高齢によって引き起こす「バランスの能力の低下」であると言う。 \cite{hoge08}


\subsection{ボディバランスの重要性}

ここで、バランス能力の重要性について考えたい。

バランス能力の向上は、高齢者の転倒事故防止だけでなく、アスリートや一般人においても効果をもたらす。

カルガリー大学が高校のバスケットボール選手と体育受講学生1000人以上を対象に、バランスボードを用いたトレーニング効果を調査したところ、バスケットボール選手ではトレーニングをしたグループの足首のケガ発生率は、しないグループよりも36%も低い、という結果が出たという。\cite{hoge09}

また医学的視点では、ボディバランスを鍛えることで、体の軸をまっすぐするために耳の中の三半規管や腹筋・背筋などトータルで鍛えることができ、反射神経や集中力といった目に見えないものを鍛えることが出来るそうだ。この集中力を鍛えることは、運動だけでなく仕事・勉強などすべての面において役に立つと医師は説明している。\cite{hoge10}

上では高齢者のバランス能力の低下を問題視したが、以上のように生活する全ての人々にとって重要な要素であることが分かる。




\subsection{既存プロダクトの問題点}

さて、話をバランスボードに戻したい。

先に様々なバランス器具の事例を挙げたが、ここで私は既存品に対する問題点を提示したい。

まず、従来のアナログ式バランスボードだが、使用方法はバランスボードの直接乗って鍛えるといういたってシンプルなものであり、誰でも簡単に使いこなす事ができる。板に乗ってバランスをとることでバランス能力を鍛える、というシンプルなプロセスであるのだが、実際自分がどのくらい上達したのか、例えばそれをリハビリテーションで行う場合、自分でバランスボードに乗った時間を図るのか、またどれだけ上達したのかを具体的に人に伝えるのは難しい。

そういう点において、フィットネスにエンタテイメント性をつけた「Wii Fit」は優れている。形状は異なるものの、今どのくらいバランスがとれたのか、その日々の成長具合をあとで自ら確認する事ができる。また、スノーボード等のビジュアル効果をつけることで娯楽感覚で楽しめるため、継続的な実行が可能であると私は考えていた。

だが、実際にWii Fitを持っているユーザによると、その継続性に難点があると言う。

以下は、実際にWii Fitを持っているユーザに対して行った、ユーザビリティにおける評価のアンケート内容と、その結果をまとめたものである。(表\ref{tb:sample01})。

\begin{table}[htbp]
\caption{Wii FItユーザに対して行ったアンケート調査}
\label{tb:sample01}
\begin{center}
\begin{tabular}{l|c|c}
\hline
質問内容&Yes&No\\\hline\hline
Q1.フィットネス目的で始めたのか.&5&0\\\hline
Q2. 効果はあったか.&1&4\\\hline
Q3. 継続したのか.&1&4\\\hline
Q4. Q3に関する主な理由.&楽しかったから.&準備するのに時間がかかる.\\\hline
 & &場所が限られるので,あまりできない.\\\hline
\end{tabular}\end{center}
\end{table}



以上の結果から、Wii Fitを立ち上げるまでの手間の煩わしさから、多くのユーザが長続きしなかったそうだ。

これを踏まえ、本研究では、そんなユーザの煩わしさを排除し、継続性を獲得できるようなバランスボードシステムを提案する。  




	% 本文2
\include{03}	% 本文3
\chapter{要素技術}
\label{chap:04}

本章では、今回の実装に用いた要素技術として、GoldFish, Lindaについて説明を行いたい.

\section{GoldFish}

GoldFishは、同研究室の橋本翔氏が開発したアプリケーション製作のためのフレームワークである。

これは、Android NFCとJavaScriptだけで実世界GUIが作れるというもので、特徴としては、

\begin{itemize}
   \item AndroidでNFCタグに触るとアプリが起動する
   \item JavaScriptでアプリケーションを書く事ができる
   \item ネイティブアプリとのブリッジがあり、JavaScriptからAndroidの各種センサにアクセスする事ができる
    \item NFCタグタッチ時にアプリケーションがロードされるため、アプリケーションをインストールしていなくても使える
 \end{itemize}
 
の4点が挙げられる。


NFC対応のAndroid端末でタグを読み込む事で、ubif.orgに登録されているリストを参照し、webページにアクセスできる仕組みになっている。

利用例として、研究室ではNFCタグをAndroid端末でタッチし、Android端末を傾けることで研究室のドアが開くというシステムが取り入れられている。これはAndroid端末をNFCタグにタッチした時にアプリケーションが開き、ジェスチャー入力でドアサーバに「open」信号を伝えることで、鍵が空く仕組みになっている。


\section{Linda}

Lindaとは、1980年代にイェール大学で生まれた並列プログラミング言語であり、JavaやC言語といった他の言語上に拡張して実装される。

タプルスペース(tuplespace)と呼ばれる共有メモリ空間に、型つきのデータレコード(タプル)を格納する。in/out/rd/inp/rdpという命令で操作することで、大抵の並列処理が記述できる。

今回は、橋本翔氏が開発したlinda-base\cite{hoge10}というサーバ環境を用い、javascriptを用いて実装を行った。

研究室には温度や照度などの様々なセンサが設置されており、常に増井研究室のlinda-base(linda.masuilab.org/delta)に情報が投げられている。(図\ref{fig:linda})

\begin{figure}[htbp]
    \begin{center}
       \fbox{\includegraphics[width=150mm]{image_linda.eps}}
    \end{center}
    \caption{linda.masuilab.org/delta}
    \label{fig:linda}
\end{figure}






	% 本文4
\chapter{実装}
\label{chap:05}

本章では、今回提案したシステムの実装・構成について述べる。

\section{システム構成}

本研究では、Android端末で傾きを検知し、それをサーバに書き込みクライアントがその情報を取得する事でリアルタイムにグラフ描画する。

そこで、以下のシステムが必要となる。
\begin{itemize}
   \item バランスボードの傾きを検知し、NFCに対応したAndroidOS端末
   \item 傾き検知し、取得した情報をlinda-baseにタプルを投げるアプリケーション
   \item 投げられたタプルを受信し、描画させるためのクライアント
\end{itemize}

使用したAndroidOS端末はSAMSUNG社製のNexusS、実装言語はJavaScriptである。

また、使用するlinda-baseは増井研究室のサーバ環境を用い(linda.masuilab.org)、本研究用にそこに「room2」というタプルスペースを作成した(linda.masuilab.org/room2)。そこにAndroid端末の傾き情報を書き込み、クライアント側はそこから情報を読み込みことで、サーバ間の送受信が可能となる。


\section{実装内容}

タプルを送信する側のサーバを実装する。

増井研究室の"linda-base"の下に"room2"というタプルスペースを作成し、接続する。


\begin{itembox}[l]{{\tt balance send.js}}
\begin{verbatim}
var io = new RocketIO().connect("http://linda.masuilab.org");
var linda = new Linda(io);
var ts = new linda.TupleSpace("room2");
io.on("connect ",function(){
  alert(io.type+"connect ");
});  
\end{verbatim}
\end{itembox}



\hspace{30mm} 

以上でlinda.masuilab.org/room2との通信が可能となる。


次に、Android端末をNFCタグにタッチした時にアプリケーションを起動させ、傾きを取得してタプルを"room2"に送信する。この時、値と一緒に"[pitch]"というタグをつける。これにより、クライアント側で"[pitch]"というタグを抽出することで傾きの値が取得できるようになる。

今回はインターバルを0.3秒に設定した。


\begin{itembox}[l]{{\tt balance send.js}}
\begin{verbatim}
setInterval(function() {
        var x = ii ;
        var ori = goldfish.orientation() ;
        pitch = ori. pitch ;
        ts.write(["pitch",y]); 
        ii++;
},300); 
\end{verbatim}
\end{itembox}

\hspace{30mm}

この"write"によって、"room2"にタプルを書き込む。

この状態を"linda.masuilab.org/room2"で確認すると、以下のようにlinda-baseの"room2"にタプルが送られている事が分かる。(図\ref{fig:08})
\hspace{30mm}

\begin{figure}[htbp]
    \begin{center}
       \fbox{\includegraphics[width=150mm]{image8.eps}}
    \end{center}
    \caption{linda-base}
    \label{fig:08}
\end{figure}




ここから、クライアント側を実装する。

先程の送信側サーバと同じく、はじめに"room2"のタプルスペースと接続し、タプルが送られて来たら"watch"して読み込むようにする。これにより、リアルタイムな送受信を行う。
この際、送る側のサーバで設定した"[pitch]"を探すことで、一緒に送られている傾きのタプルを読み込める様になっている。


\begin{itembox}[l]{{\tt balance host.js}}
\begin{verbatim}
var xx = iii ;
     ts. watch(["pitch"],function(tuple){
            var k = tuple[1];
            data.addRow([iii, k, null, null, null]);
            chart.draw(data, options);
            iii++;
    });
\end{verbatim}
\end{itembox}


\hspace{30mm} 

あとは読みこんだ値を描画するだけである。今回はリアルタイムにグラフ化を行った。

この時、「GoogleChartTool」というGoogleが提供しているグラフ用ライブラリを使用した。

以下がその結果である。(図\ref{fig:10})
\hspace{30mm}

\begin{figure}[htbp]
    \begin{center}
       \fbox{\includegraphics[width=150mm]{image10.eps}}
    \end{center}
    \caption{linda-base}
    \label{fig:10}
\end{figure}







	% 本文4
\chapter{考察}
\label{chap:06}


本研究では、ユーザビリティにおける煩わしさを解消し、継続性を獲得できるようなバランスボードシステムの提案を目的として進めてきた。

それを達成するために、バランスボードに取り付けられたAndroid端末にNFCタグをタッチするだけで、リアルタイムなグラフを描画するよう設計した。これにより、Wii Fitに対してユーザが課題視していたセットアップの煩わしさはなく、誰でも簡単にはじめることができる。

実装においては、GoldFishやLinda_baseを使用したことで、プログラミングを得意としていなくても、簡単なJavaScriptのプログラムでクライアントまで実装することができた。

今回は値をリアルタイムにグラフ化する設計のみとなったが、クライアント側のデザインは自由なので、これを応用する事でさまざまなアプリケーションを製作することができる。

例えば傾きによってキャラクターがスノーボードをしているよう設計することで、Wii Fitが提供しているようなエンタテイメント性があるシステムを作る事も可能である。

また、今回はlinda-baseというサーバ環境を使用したが、同じタプルスペースと接続する事で、遠隔地からでも複数のクライアントからリアルタイムに情報を取得する事ができ、例えば遠く離れた友人同士でもリアルタイムに同じゲーム画面を共有し、楽しむ事が出来る。

その他にも、今回はリアルタイムで値をグラフ描画する設計であったが、取得した情報をデータベースにストックできるようにすることで、例えばリハビリテーションを行っている患者の医師が、後でその患者の日々の結果を、遠隔地からでもチェックすることができる。

人々の運動・健康意識が高まる現在、その需要に応えるかのように様々なサービス・システムが開発されているが、問題視されている子供の運動能力の低下や高齢者の事故増加が、こうした技術が人々の行動にアフォードし、これらの問題が改善される日が来るのを期待している。

	% 本文4

\include{90_acknowledgment}	% 謝辞。要独自コマンド、include先参照のこと
\include{91_bibliography}	% 参考文献。要独自コマンド、include先参照のこと
\appendix
\include{92_appendix}		% 付録

\end{document}

