\chapter{バランスボードの意義}
\label{chap:02}

本章では、バランスボードの歴史と需要、また関連したプロダクトの紹介や、既存のバランスボードが抱える課題点について述べていきたい。


\section{歴史}

元々バランスボートというのは、ライフル射撃のナショナルチームでのトレーニングに採用されたのがきっかけだという。それをきっかけに、怪我の軽減や運動パフォーマンスの向上、リハビリやメンタルトレーニングといった面で各競技において効果が表れ、少しずつ注目されていった。 \cite{hoge03}

当初のタイプは縦長の板の裏側に支点となる突起がついた、いわゆるシンプルな型であったというが、現在は用途に応じて様々なタイプのものがある。それを下で紹介する。


\section{種類}

ここでは、現在あるバランスボードの種類をいくつか紹介していきたい。

\subsection{突起型バランスボード}

これが従来のタイプのものである。当初と同じ、板の裏に支点となる突起がついたもの(図\ref{fig:01})や、縦長の板状で支点となる突起が棒状になっていて転がしてバランスをとるタイプのもの(図\ref{fig:02})などがある。



\begin{figure}[htbp]
 \begin{minipage}{0.5\hsize}
  \begin{center}
   \fbox{\includegraphics[height=30mm]{image02.eps}}
  \end{center}
  \caption{タイプ}
  \label{fig:01}
 \end{minipage}
 \begin{minipage}{0.5\hsize}
  \begin{center}
   \fbox{\includegraphics[height=30mm]{image01.eps}}
  \end{center}
  \caption{縦長タイプ}
  \label{fig:02}
 \end{minipage}
\end{figure}



これらはいわゆる従来型のバランスボードで、医療機関でのリハビリテーションやスポーツ選手のトレーニングなどでよく使用される。

また、ボード状のものだけではなく、球状のタイプでその上に乗ってバランスをとるバランスボールや、不安定な平型のクッション状のものにのってバランスをとるものもある。


\subsection{バランスWiiボード}

これが先に例で挙げてきた、任天堂株式会社が開発した「Wii FIt」を使用する際に用いるものである。(図\ref{fig:03})


\begin{figure}[htbp]
    \begin{center}
       \fbox{\includegraphics[height=30mm]{image03.eps}}
    \end{center}
    \caption{バランスWiiボード}
    \label{fig:03}
\end{figure}



従来のバランスボードとは異なり、板を支える支点となる突起や棒はなく、水平な板状の型をしており、ボードに内蔵されたセンサがその板上でのユーザの体重移動を検知することで、ユーザのバランス状況を解析する仕組みである。ユーザはこの水平なボードの上に乗り、画面上で繰り広げられる綱渡りやスノーボードといった娯楽を、まるで実際に行っているかのような感覚で体重移動することで、画面上の自分が動いて楽しめる、という仕様になっている。

そうしたスポーツ娯楽を始め、他にもヨガやピラティスといった本格的なフィットネスも楽しめ、今では40種類以上のフィットネスを楽しむ事ができるという。 \cite{hoge04}



\subsection{トーニングシューズ}

バランスボードとは種類が少し異なるが、これはバランスボールと同じ効果が得られるようデザインされたもので、靴底や中敷きの形状・素材に工夫がなされている。通常のシューズより足への負荷を高められ、エクササイズ目的で開発された靴である。(図\ref{fig:04})


\begin{figure}[htbp]
    \begin{center}
       \fbox{\includegraphics[height=30mm]{image04.eps}}
    \end{center}
    \caption{トーニングシューズ}
    \label{fig:04}
\end{figure}


上の写真の靴(LAギア ウォーキングトーン)は、靴底が接地面積が小さくなるよう設計されており、左右方向はやや不安定になり、意識的に体幹でバランスをとるようデザインされたものである。 \cite{hoge05}

現在、様々なメーカーでこういったフィットネス効果目的の商品が開発されている。




\section{需要}

ここでは、社会的な需要や既存プロダクトの問題点などについて、実際に行ったアンケート調査をふまえて述べていきたい。

\subsection{社会的現状}

先に挙げた健康・運動意識調査にもあったように、特に近年の高齢者におけるフィットネスの意識は高く、子供からお年寄りまで幅広い世代のユーザをターゲットにした「Wii FIt」が注目されているという点からも、社会全体としてこうした運動・健康傾向が表れている印象を受ける。これ以外にも、トーニングシューズなどフィットネス効果を狙った様々なプロダクト・サービスが開発されているのも、人々の関心の高さの象徴であろう。

その一方で、社会的問題として、子供の運動能力の低下や高齢者の公共空間における事故の増加が挙げられる。

文部科学省が行った「体力・運動能力調査」によると、子供の体力・運動能力の低下は昭和60年ごろから現在まで低下傾向が続いており、その原因として、保護者をはじめとする国民の意識の中で、外遊びやスポーツの重要性を学力の状況と比べ軽視する傾向が進んだ事にあるとされている。\cite{hoge06} さらに、テレビゲームやポータブルゲームの流行により、子供達が外で遊ぶより室内で遊ぶことが多くなった現状もあげられる。

また、高齢者の事故の増加に関しては、消防庁が行った調査の統計結果によると、高齢者の事故がここ5年間で1万人以上増加しており、その約8割を占める原因が「転倒事故」だという。 \cite{hoge07}そして、これら転倒事故が起こってしまう要因として主にあげられるのが、高齢によって引き起こす「バランスの能力の低下」であると言う。 \cite{hoge08}


\subsection{ボディバランスの重要性}

ここで、バランス能力の重要性について考えたい。

バランス能力の向上は、高齢者の転倒事故防止だけでなく、アスリートや一般人においても効果をもたらす。

カルガリー大学が高校のバスケットボール選手と体育受講学生1000人以上を対象に、バランスボードを用いたトレーニング効果を調査したところ、バスケットボール選手ではトレーニングをしたグループの足首のケガ発生率は、しないグループよりも36%も低い、という結果が出たという。\cite{hoge09}

また医学的視点では、ボディバランスを鍛えることで、体の軸をまっすぐするために耳の中の三半規管や腹筋・背筋などトータルで鍛えることができ、反射神経や集中力といった目に見えないものを鍛えることが出来るそうだ。この集中力を鍛えることは、運動だけでなく仕事・勉強などすべての面において役に立つと医師は説明している。\cite{hoge10}

上では高齢者のバランス能力の低下を問題視したが、以上のように生活する全ての人々にとって重要な要素であることが分かる。




\subsection{既存プロダクトの問題点}

さて、話をバランスボードに戻したい。

先に様々なバランス器具の事例を挙げたが、ここで私は既存品に対する問題点を提示したい。

まず、従来のアナログ式バランスボードだが、使用方法はバランスボードの直接乗って鍛えるといういたってシンプルなものであり、誰でも簡単に使いこなす事ができる。板に乗ってバランスをとることでバランス能力を鍛える、というシンプルなプロセスであるのだが、実際自分がどのくらい上達したのか、例えばそれをリハビリテーションで行う場合、自分でバランスボードに乗った時間を図るのか、またどれだけ上達したのかを具体的に人に伝えるのは難しい。

そういう点において、フィットネスにエンタテイメント性をつけた「Wii Fit」は優れている。形状は異なるものの、今どのくらいバランスがとれたのか、その日々の成長具合をあとで自ら確認する事ができる。また、スノーボード等のビジュアル効果をつけることで娯楽感覚で楽しめるため、継続的な実行が可能であると私は考えていた。

だが、実際にWii Fitを持っているユーザによると、その継続性に難点があると言う。

以下は、実際にWii Fitを持っているユーザに対して行った、ユーザビリティにおける評価のアンケート内容と、その結果をまとめたものである。(表\ref{tb:sample01})。

\begin{table}[htbp]
\caption{Wii FItユーザに対して行ったアンケート調査}
\label{tb:sample01}
\begin{center}
\begin{tabular}{l|c|c}
\hline
質問内容&Yes&No\\\hline\hline
Q1.フィットネス目的で始めたのか.&5&0\\\hline
Q2. 効果はあったか.&1&4\\\hline
Q3. 継続したのか.&1&4\\\hline
Q4. Q3に関する主な理由.&楽しかったから.&準備するのに時間がかかる.\\\hline
 & &場所が限られるので,あまりできない.\\\hline
\end{tabular}\end{center}
\end{table}



以上の結果から、Wii Fitを立ち上げるまでの手間の煩わしさから、多くのユーザが長続きしなかったそうだ。

これを踏まえ、本研究では、そんなユーザの煩わしさを排除し、継続性を獲得できるようなバランスボードシステムを提案する。  




