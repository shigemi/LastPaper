\chapter{考察}
\label{chap:06}


本研究では、ユーザビリティにおける煩わしさを解消し、継続性を獲得できるようなバランスボードシステムの提案を目的として進めてきた。

それを達成するために、バランスボードに取り付けられたAndroid端末にNFCタグをタッチするだけで、リアルタイムなグラフを描画するよう設計した。これにより、Wii Fitに対してユーザが課題視していたセットアップの煩わしさはなく、誰でも簡単にはじめることができる。

実装においては、GoldFishやLinda_baseを使用したことで、プログラミングを得意としていなくても、簡単なJavaScriptのプログラムでクライアントまで実装することができた。

今回は値をリアルタイムにグラフ化する設計のみとなったが、クライアント側のデザインは自由なので、これを応用する事でさまざまなアプリケーションを製作することができる。

例えば傾きによってキャラクターがスノーボードをしているよう設計することで、Wii Fitが提供しているようなエンタテイメント性があるシステムを作る事も可能である。

また、今回はlinda-baseというサーバ環境を使用したが、同じタプルスペースと接続する事で、遠隔地からでも複数のクライアントからリアルタイムに情報を取得する事ができ、例えば遠く離れた友人同士でもリアルタイムに同じゲーム画面を共有し、楽しむ事が出来る。

その他にも、今回はリアルタイムで値をグラフ描画する設計であったが、取得した情報をデータベースにストックできるようにすることで、例えばリハビリテーションを行っている患者の医師が、後でその患者の日々の結果を、遠隔地からでもチェックすることができる。

人々の運動・健康意識が高まる現在、その需要に応えるかのように様々なサービス・システムが開発されているが、問題視されている子供の運動能力の低下や高齢者の事故増加が、こうした技術が人々の行動にアフォードし、これらの問題が改善される日が来るのを期待している。

