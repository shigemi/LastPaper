\chapter{序論}
\label{chap:introduction}


\section{研究背景}

近年、ICT技術の発展に伴い「ユビキタス社会」が人々にますます浸透してきている。
先に挙げたスマートフォン端末の例だと、たった指先程度の大きさしかない数々のセンサが、人々の生活の手間は省き、生活における利便性を高めた。
スマートフォン以外の事例では、駅の改札にてカードを改札にタッチするだけで構内に入れ、オートチャージ機能により、かつての切符を買う手間を要さなくなった。子供の玩具においては、かつてはゲーム機を手に持ち、起用に指でコントローラーを操作して遊んでいたものを、今やラケットを持つかの様にゲーム機を握り、テレビの前で本当にテニスや卓球を行っているかのようにができるようになった。

これを実現させたものが、任天堂株式会社が開発した「Wii」シリーズである。中でも、「Wii Fit」というバランスボード型のタイプのものは、「ゲーム感覚でフィットネスをする」という新しい概念のもと生み出されてたものであり、この「ゲーム性とフィットネス性を兼ねた」新しいエンタテイメントとして、最近注目されている。

こうした背景には、近年、国民の運動・健康に対する意識が高まっている現状があるだろう。厚生労働省が行った2013年の国民健康・栄養調査によると、健康づくりのための身体活動や運動を実践している者の割合は39.4%であり、70歳位上では半数を超えているそうだ。特に、70歳以上では、男女とも「要介護とならないため、もしくは悪化させないため」と回答した者の割合が最も多かった。\cite{hoge02}

こうした健康志向の高まっている高齢者も顧客ターゲットに含めた事で、今までのテレビゲームになかった「子供からお年寄りまでの幅広い層のユーザ獲得」につながったという。

その一方で、ユーザビリティの問題点から、なかなか継続するのが困難であるという意見もある。これに関しては、行ったアンケート調査結果を後の章で紹介するが、Wii Fitをするためのセットアップに手間がかかるなど、ユーザからは様々な指摘が挙げられた。




\section{研究目的}

本研究の目的は、以上のユーザビリティの煩わしさをセンサを用いて解決し、高まる人々の健康意識にあやかって開発された「Wii Fit」のフューチャーワークとしても用いる事ができるシステムを提案することにある。

今回は、スマートフォン端末を使用したバランスボードシステムを実装した。



\section{本論分について}

本論分は以下のように構成されている。

第\ref{chap:02}章では、バランスボードについて述べる。バランスボードの需要や役割、関連したプロダクトの紹介や問題点などについてまとめる。

%第\ref{chap:03}章では、関連研究について述べる。

第\ref{chap:04}章では、要素技術について述べる。実装で使用したシステム・技術について説明する。

第\ref{chap:05}章では、実装内容について述べる。システム構成や実装方法、また用途例を紹介する。

第\ref{chap:06}章では、最後に本研究のまとめと、今後の応用例などを考察したい。

また最後に、付録として実装のソースコードを貼付ける。
